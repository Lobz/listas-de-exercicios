\documentclass{article}
 \usepackage[brazil]{babel}
 \usepackage[utf8]{inputenc}
 \usepackage{amsmath}
 \usepackage{amssymb}
 \usepackage{amsthm}
 \usepackage{exercise}
 \def\ExerciseListName{Ex.}%
 \def\ExerciseName{Ex.}%
 \usepackage{multicol}
\usepackage[left=3.5cm,top=3.0cm,right=4.0cm]{geometry}


\title{Vetores, Sistemas de Equações e Espaços Vetoriais}
\author{Mali}

\begin{document}
\maketitle


\begin{enumerate}

		\item Para cada sistema de equações abaixo,
		\begin{itemize}
			\item classifique o sistema em homogêneo ou não homogêneo
			\item escreva a equação matricial correspondente
			\item classifique a matriz em LI ou LD
			\item classifique a solução (solução única, solução indeterminada ou sem solução)
			\item se a solução for única, apresente a solução
			\item se a solução for indeterminada, apresente uma forma geral da solução
			\item usando o R, calcule o determinante da matriz
		\end{itemize}
		(obs.: faça os ítens na ordem que preferir)
		\begin{enumerate}
			\item (exemplo) \begin{align}
				2x + y &= 3\\
				x -y &= 5
			\end{align}
(solução) 
		\begin{itemize}
		\item não-homogêneo
		\item forma matricial: \[ \left( \begin{array}{ccc}
			2 & 1 \\
			1 & -1 \end{array} \right)\left( \begin{array}{ccc} x\\y \end{array} \right) 
			= \left( \begin{array}{ccc} 3\\5 \end{array} \right)\] 
		\item LI
		\item Solução única
		\item $x = 8/3$, $y=-7/3$
		\item determinante $-2-1=-3$
		\end{itemize}
		\begin{multicols}{2}
			\item \begin{align}
				2x + 2y &= 0\\
				x -y &= 0
			\end{align}
			\item \begin{align}
				2x + 2y + 2z &= 0\\
				x -y + z &= 0\\
				x -3z &=0
			\end{align}
			\item \begin{align}
				2x + 2y + 2z &= 0\\
				x -y + z &= 5\\
				x +5y +z &=0
			\end{align}
			\item \begin{align}
				x -y &= 0\\
				x +y -z &= 0\\
				x&=5
			\end{align}
			\item \begin{align}
				2x + 2y &= 0\\
				x -y &= 0
			\end{align}
			\item \begin{align}
				2x + 2y &= 0\\
				x -y &= 0
			\end{align}
		\end{multicols}
		\end{enumerate}

\end{enumerate}
	\end{document}
