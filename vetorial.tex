\documentclass{article}
 \usepackage[brazil]{babel}
 \usepackage[utf8]{inputenc}
 \usepackage{amsmath}
 \usepackage{amssymb}
 \usepackage{amsthm}
 \usepackage{exercise}
 \def\ExerciseListName{Ex.}%
 \def\ExerciseName{Ex.}%
 \usepackage{multicol}
\usepackage[left=3.5cm,top=3.0cm,right=4.0cm]{geometry}


\title{Vetores, Sistemas de Equações e Espaços Vetoriais}
\author{Mali}

\begin{document}
\maketitle


\begin{enumerate}

		\item Para cada sistema de equações abaixo,
		\begin{itemize}
			\item classifique o sistema em homogêneo ou não homogêneo
			\item escreva a equação matricial correspondente
			\item classifique as linhas da matriz em LI ou LD
			\item classifique a solução (solução única, solução indeterminada ou sem solução)
			\item se a solução for única, apresente a solução
			\item se a solução for indeterminada, apresente uma forma geral da solução
			\item calcule o determinante da matriz
		\end{itemize}
		(obs.: faça os ítens na ordem que preferir)
			\item (exemplo) \begin{align*}
				2x + y &= 3\\
				x -y &= 5
			\end{align*}
(solução) 
		\begin{itemize}
		\item não-homogêneo
		\item forma matricial: \[ \left( \begin{array}{ccc}
			2 & 1 \\
			1 & -1 \end{array} \right)\left( \begin{array}{ccc} x\\y \end{array} \right) 
			= \left( \begin{array}{ccc} 3\\5 \end{array} \right)\] 
		\item LI
		\item Solução única
		\item $x = 8/3$, $y=-7/3$
		\item determinante $-2-1=-3$
		\end{itemize}
		\begin{enumerate}
		\begin{multicols}{2}
			\item \begin{align*}
				2x + 2y &= 0\\
				x -y &= 0
			\end{align*}
			\item \begin{align*}
				2x + 2y + 2z &= 0\\
				x -y + z &= 0\\
				x -3z &=0
			\end{align*}
			\item \begin{align*}
				2x + 2y + 2z &= 0\\
				x -y + z &= 5\\
				x +5y +z &=0
			\end{align*}
			\item \begin{align*}
				x -y &= 0\\
				x +y -z &= 0\\
				x&=5
			\end{align*}
			\item \begin{align*}
				x + 3y &= 5\\
				x -y &= 2\\
				5x+y/2&=10
			\end{align*}
			\item \begin{align*}
				2y +z&= 0\\
				x -y -z&= 0
			\end{align*}
		\end{multicols}
		\end{enumerate}
	
	\item Nos casos abaixo, a equação está na forma matricial. Escreva o sistema de equações
		correspondente, e então faça os mesmos passos que no exercício anterior.
		\begin{enumerate}
		\begin{multicols}{2}
		\item \[ \left( \begin{array}{ccc}
			1 & 1 \\
			1 & -1 \end{array} \right)\left( \begin{array}{ccc} x\\y \end{array} \right) 
			= \left( \begin{array}{ccc} 1\\2 \end{array} \right)\]
		\item \[ \left( \begin{array}{ccc}
				1 & 1\\
			1 & 0 \end{array} \right)\left( \begin{array}{ccc} x\\y \end{array} \right) 
			= \left( \begin{array}{ccc} 0\\0 \end{array} \right)\]
		\item \[ \left( \begin{array}{ccc}
				1 & 1 & 0\\
			1 & 0 & 1\end{array} \right)\left( \begin{array}{ccc} x\\y\\z \end{array} \right) 
			= \left( \begin{array}{ccc} 0\\0\\2 \end{array} \right)\]
		\item \[ \left( \begin{array}{ccc}
				1 & 1 & 0\\
				1 & 2 & 0\\
			1 & 0 & 0 \end{array} \right)\left( \begin{array}{ccc} x\\y\\z \end{array} \right) 
			= \left( \begin{array}{ccc} 2\\5\\4 \end{array} \right)\]
		\item \[ \left( \begin{array}{ccc}
				2 & 1 & 2\\
				1 & 2 & 1\\
			0 & 0 & 0 \end{array} \right)\left( \begin{array}{ccc} x\\y\\z \end{array} \right) 
			= \left( \begin{array}{ccc} 0\\0\\0 \end{array} \right)\]
		\item \[ \left( \begin{array}{ccc}
				1 & 1 & 1\\
				1 & 2 & 2\\
			1 & 0 & 3 \end{array} \right)\left( \begin{array}{ccc} x\\y\\z \end{array} \right) 
			= \left( \begin{array}{ccc} 2\\1\\1 \end{array} \right)\]
		\end{multicols}
		\end{enumerate}
	
	\item Quais dos seguintes conjuntos são bases do espaço $\mathbb{R}^3$? Tente desenhar o
		espaço gerado por esses conjuntos.
		\begin{enumerate}
		\begin{multicols}{2}
		\item \[ \left\{ \left( \begin{array}{ccc} 1\\2\\3 \end{array} \right) , 
				\left( \begin{array}{ccc} 2\\0\\5 \end{array} \right) \right\}\]
		\item \[ \left\{ \left( \begin{array}{ccc} 1\\2\\3 \end{array} \right) , 
				\left( \begin{array}{ccc} 2\\2\\3 \end{array} \right) , 
					\left( \begin{array}{ccc} 2\\0\\5 \end{array} \right)
						\right\}\]
		\item \[ \left\{ \left( \begin{array}{ccc} 1\\2\\3 \end{array} \right) , 
				\left( \begin{array}{ccc} 2\\2\\3 \end{array} \right) , 
				\left( \begin{array}{ccc} 4\\5\\6 \end{array} \right) , 
					\left( \begin{array}{ccc} 2\\0\\5 \end{array}
						\right)\right\}\]
		\item \[ \left\{ \left( \begin{array}{ccc} 0\\0\\0 \end{array} \right) , 
				\left( \begin{array}{ccc} 2\\5\\6 \end{array} \right) , 
					\left( \begin{array}{ccc} 4\\2\\4 \end{array} \right) \right\}\]
		\item \[ \left\{ \left( \begin{array}{ccc} 1\\2\\3 \end{array} \right) , 
				\left( \begin{array}{ccc} 2\\4\\6 \end{array} \right) , 
					\left( \begin{array}{ccc} 2\\10\\15 \end{array} \right) \right\}\]
		\item \[ \left\{ \left( \begin{array}{ccc} 1\\2\\-3 \end{array} \right) , 
				\left( \begin{array}{ccc} 2\\4\\6 \end{array} \right) , 
					\left( \begin{array}{ccc} 2\\-3\\5 \end{array} \right) \right\}\]
		\end{multicols}
		\end{enumerate}
	
	\item Para as seguintes bases do espaço $\mathbb{R}^2$, escreva as matrizes de mudança de
		base para a base canônica e vice-versa.
		\begin{enumerate}
		\begin{multicols}{2}
		\item \[ \left\{ \left( \begin{array}{ccc} 1\\2 \end{array} \right) , 
				\left( \begin{array}{ccc} 2\\0 \end{array} \right) \right\}\]
		\item \[ \left\{ \left( \begin{array}{ccc} 1\\1 \end{array} \right) , 
				\left( \begin{array}{ccc} 1\\0 \end{array} \right) \right\}\]
		\item \[ \left\{ \left( \begin{array}{ccc} 1\\1 \end{array} \right) , 
				\left( \begin{array}{ccc} 1\\-1 \end{array} \right) \right\}\]
		\item \[ \left\{ \left( \begin{array}{ccc} 0\\1 \end{array} \right) , 
				\left( \begin{array}{ccc} 1\\0 \end{array} \right) \right\}\]
		\item \[ \left\{ \left( \begin{array}{ccc} 3\\-1 \end{array} \right) , 
				\left( \begin{array}{ccc} 1\\-2 \end{array} \right) \right\}\]
		\item \[ \left\{ \left( \begin{array}{ccc} -1\\-1 \end{array} \right) , 
				\left( \begin{array}{ccc} -1\\-2 \end{array} \right) \right\}\]
		\end{multicols}
		\end{enumerate}
	
	\item Vamos verificar o efeito da multiplicação de algumas matrizes por uma matriz genérica, 
		\[M = \left( \begin{array}{ccc} a & b & c\\d & e & f\\g & h & i \end{array}
			\right)\]. Calcule os produtos:
		\begin{enumerate}
		\begin{multicols}{2}
		\item \[M\left( \begin{array}{ccc} 2 & 0 & 0\\0 & 1 & 0\\0 & 0 & 1 \end{array} \right)\]
		\item \[\left( \begin{array}{ccc} 2 & 0 & 0\\0 & 1 & 0\\0 & 0 & 1 \end{array} \right)M\]
		\item \[M\left( \begin{array}{ccc} 1 & 0 & 0\\0 & 3 & 0\\0 & 0 & 1 \end{array} \right)\]
		\item \[\left( \begin{array}{ccc} 1 & 0 & 0\\0 & 3 & 0\\0 & 0 & 1 \end{array} \right)M\]
		\item \[M\left( \begin{array}{ccc} 2 & 0 & 0\\0 & 3 & 0\\0 & 0 & 1 \end{array} \right)\]
		\item \[\left( \begin{array}{ccc} 2 & 0 & 0\\0 & 3 & 0\\0 & 0 & 1 \end{array} \right)M\]
		\item \[M\left( \begin{array}{ccc} 0 & 1 & 0\\1 & 0 & 0\\0 & 0 & 1 \end{array} \right)\]
		\item \[\left( \begin{array}{ccc} 0 & 1 & 0\\1 & 0 & 0\\0 & 0 & 1 \end{array}
				\right)M\]
		\item \[M\left( \begin{array}{ccc} 0 & 0 & 1\\1 & 0 & 0\\0 & 1 & 0 \end{array} \right)\]
		\item \[\left( \begin{array}{ccc} 0 & 0 & 1\\1 & 0 & 0\\0 & 1 & 0 \end{array} \right)M\]
		\item \[M\left( \begin{array}{ccc} 1 & 0 & 0\\1 & 1 & 0\\0 & 0 & 1 \end{array} \right)\]
		\item \[\left( \begin{array}{ccc} 1 & 0 & 0\\1 & 1 & 0\\0 & 0 & 1 \end{array} \right)M\]
		\item \[M\left( \begin{array}{ccc} 1 & 0 & 0\\0 & 1 & 0\\0 & 1 & 1 \end{array} \right)\]
		\item \[\left( \begin{array}{ccc} 1 & 0 & 0\\0 & 1 & 0\\0 & 1 & 1 \end{array} \right)M\]
		\item \[M\left( \begin{array}{ccc} 1 & 0 & 0\\0 & 1 & 0\\0 & -1 & 1 \end{array} \right)\]
		\item \[\left( \begin{array}{ccc} 1 & 0 & 0\\0 & 1 & 0\\0 & -1 & 1 \end{array} \right)M\]
		\item \[M\left( \begin{array}{ccc} 1 & 0 & 0\\1 & 1 & 0\\1 & 1 & 1 \end{array} \right)\]
		\item \[\left( \begin{array}{ccc} 1 & 0 & 0\\1 & 1 & 0\\1 & 1 & 1 \end{array} \right)M\]
		\end{multicols}
		\end{enumerate}

	\item Como você descreveria os efeitos das multiplicações acima sobre a matriz M? Usando
		essa intuição, tente encontrar as inversas dessas matrizes. Lembre que se
		$AM=N$, então $A^{-1}N=M$, ou seja, a matriz inversa desfaz a transformação
		original.

	\item Mostre que o produto de duas matrizes inversíveis é uma matriz inversível.
	
			


\end{enumerate}
	\end{document}
