\documentclass{article}
 \usepackage[brazil]{babel}
 \usepackage[utf8]{inputenc}
 \usepackage{amsmath}
 \usepackage{amssymb}
 \usepackage{amsthm}
 \usepackage{exercise}
 \def\ExerciseListName{Ex.}%
 \def\ExerciseName{Ex.}%
 \usepackage{multicol}
\usepackage[left=3.5cm,top=3.0cm,right=4.0cm]{geometry}


\title{Logaritmos e Potências}
\author{Mali}

\begin{document}
\maketitle

Definição: logaritmo de um número é a potência à qual preciso elevar a base para chegar a esse
número. Então se $log_b(n) = p$, isso significa que $b^p = n$.

\begin{enumerate}

	\item Calcule as potências e os logaritmos:
		\begin{enumerate}
		\begin{multicols}{2}
			\item $2^1=$
			\item $2^2=$
			\item $2^3=$
			\item $2^4=$
			\item $log_2(2)=$
			\item $log_2(4)=$
			\item $log_2(8)=$
			\item $log_2(16)=$
		\end{multicols}
		\end{enumerate}
	
		Lembre-se que potência com expoente negativo é o inverso da potência com expoente
		positivo, isto é: $b^{-n} = 1/b^n$.
	\item Calcule as potências e os logaritmos:
		\begin{enumerate}
		\begin{multicols}{3}
		\item $3^{-1}=$
		\item $4^{-2}=$
		\item $10^{-3}=$
		\item $5^{-3}=$
			\item $log_3(1/3)=$
			\item $log_4(1/16)=$
			\item $log_{10}(1/1000)=$
			\item $log_5(1/125)=$
			\item $log_3(1/9)=$
			\item $log_4(1/4)=$
			\item $log_{10}(0,001)=$
			\item $log_5(1/525)=$
		\end{multicols}
		\end{enumerate}
	
		Quando multiplicamos potências com a mesma base, podemos somar os expoentes
		($b^nb^m=b^{m+n}$). O correspondente dessa regra em logaritmos é que a soma de
		logaritmos (expoentes) equivale a uma multiplicação.	
	\item Calcule as potências e os logaritmos:
		\begin{enumerate}
		\begin{multicols}{2}
		\item $2^32^2=$
		\item $4^{-2}4^2=$
		\item $10^{-3}10^{-1}=$
		\item $5^{3}5^{-2}=$
		\item $e^{1}e^{-2}=$
		\item $2^5=$
		\item $4^0=$
		\item $10^{-4}=$
		\item $5^{1}=$
		\item $e^{-1}=$
		\item $log_2(8)+log_2(4)=$
		\item $log_4(1/16)+log_4(16)=$
		\item $log_{10}(0,001)+log_10(0,1)=$
		\item $log_5(125)+log_5(1/25)=$
		\item $ln(e)+ln(1/e^2)=$
		\item $log_2(8x4)=$
		\item $log_4(16/16)=$
		\item $log_{10}(0,0001)=$
		\item $log_5(5)=$
		\item $ln(1/e)=$
		\end{multicols}
		\end{enumerate}
	
		O logaritmo é uma operação inversa à exponenciação, no sentido de que $log_b(b^x)=x$
		e $b^{log_b(x)}=x$. Em particular, $x=e^{ln(x)}$.
	\item Coloque as fórmulas a seguir na forma $e^{ax+b}$:
		\begin{enumerate}
		\begin{multicols}{2}
		\item $2=e^{ln(2)}$
		\item $2^x=$
		\item $3e^x=$
		\item $4e^{2x}=$
		\item $2\pi2^x=$
		\item $3^x2^x=$
		\end{multicols}
		\end{enumerate}
	
	\item Imagine um conjunto de dados com uma distribuição com a forma $y_i =
		e^{2x_i+1+\epsilon_i}$ onde $epsilon \sim N(0,2)$ é um erro normalmente distribuído - ou seja,
		em escala logarítmica, o erro é normal e a variância não depende de $x$.
		\begin{enumerate}
			\item Suponha que $x_1=1$ e que o erro $\epsilon_1$ vale 2. Quanto vale
				$y_1$? Qual a diferença entre o $y$ esperado para esse $x_1$ e o
				$y_1$ encontrado com esse erro? (obs.: use calculadora)
			\item Suponha agora que $x_2=3$ e que o erro $\epsilon_2$ também vale 2. Quanto vale
				$y_2$? Qual a diferença entre o $y$ esperado para esse $x_2$ e o
				$y_2$ encontrado com esse erro?
		\end{enumerate}

\end{enumerate}
	\end{document}
