\documentclass{article}
 \usepackage[brazil]{babel}
 \usepackage[utf8]{inputenc}
 \usepackage{amsmath}
 \usepackage{amssymb}
 \usepackage{amsthm}
 \usepackage{exercise}
 \usepackage{multicol}
 \def\ExerciseListName{Ex.}%
 \def\ExerciseName{Ex.}%
\usepackage[left=3.5cm,top=3.0cm,right=4.0cm]{geometry}


\title{Lista 2 - Módulo e Equações}
\author{Mali - Álgebra}

\begin{document}
\maketitle

Resolver uma equação significa a) identificar qual a incógnita e b) encontrar todos os valores possíveis para a incógnita que fazem a igualdade ser verdade.
Por exemplo, resolver a equação 
\begin{equation}
	x^2 = 4
\end{equation}
significa se perguntar "quais são os números que elevados ao quadrados dão 4?". Sabemos que $2^2 = 4$, e que $(-2)^2 = 4$. Então 2 e $-2$ são soluções da equação.
Existe alguma outra solução? Bem, se um número $y$ é maior que 2 ou menor que $-2$, então $y^2 > 4$; por outro lado, se $y$ está entre $-2$ e 2, então $y^2 < 4$. Com
isso, sabemos que nenhum outro número real é solução dessa equação -- as duas únicas soluções são 2 e $-2$.


\begin{enumerate}

		\item Some as frações, colocando o resultado sobre um único denominador
		\begin{enumerate}
		\begin{multicols}{2}
			\item exemplo: $\frac{y}{2} + \frac{1}{4} = \frac{2y + 1}{4} $
			\item $\frac{1}{3} + \frac{2}{7}$
			\item $1 + \frac{1}{3}$
			\item $3 + \frac{1}{4} - \frac{2}{5}$
			\item $x + \frac{x}{2}$
			\item $x + \frac{1}{x}$
			\item $\frac{1}{x} + \frac{1}{3x}$
			\item $\frac{1}{4x} - \frac{1}{6x}$
			\item $\frac{x}{3} + \frac{1}{2} - x$
			\item $\frac{a}{2} + a - \frac{1}{a}$
			\item $\frac{a}{2} + \frac{2}{10} + \frac{1}{a}$
			\item $n(n-1) - \frac{1}{2}$
		\end{multicols}
		\end{enumerate}

		\item Resolva as equações:
		\begin{enumerate}
		\begin{multicols}{2}
			\item $x+5=10$
			\item $2x -3=2-x$
			\item $5y = 2-y$
			\item $\frac{y}{3} + \frac{y}{6} = y + 2$
			\item $\frac{1}{t} = \frac{2}{t} + \frac{1}{2}$
			\item $\frac{1}{3} + t = 1$
		\end{multicols}
		\end{enumerate}

		\item O módulo de um número, ou valor absoluto do número, é a distância dele até o zero, ou
            seja, na prática, o valor dele sem o sinal. Assim $|2| = |-2| = 2$. Calcule:
		\begin{enumerate}
		\begin{multicols}{2}
			\item $|3| = $
			\item $|-4| = $
			\item $|1-2| =$
			\item $|(-3)^3|=$
		\end{multicols}
		\end{enumerate}

		\item Podemos escrever o valor do módulo como:
            \[ |x| = \left\{ \begin{array}{ccc}
                x, \, x \geq 0,\\
            -x, \, x < 0 \end{array}\]
        Analogamente, quando temos o módulo de uma expressão, podemos dividir em casos como a
        seguir:
                \[ |x-1| = \left\{ \begin{array}{ccc}
                x-1, \, x \geq 1,\\
                1-x, \, x < 1
                \end{array} \]
        Faça essa separação em casos para as seguintes expressões:
		\begin{enumerate}
		% first column
		\begin{multicols}{2}
			\item $|-x| = $
			\item $|2-x| = $
			\item $|-1-x| = $
			\item $|-1+a| = $
			\item $|2a+3| = $
			\item $|3b-3| =$
			\item $|b^2+8|=$
			\item $|6-b^2|=$
			\item $|4+2b^2|=$
			\item $|3-b^3|=$
		\end{multicols}
		\end{enumerate}
		
		\item Resolva as seguintes equações envolvendo módulos, encontrando todas as soluções. Dica: quando em dúvida, use a
            técnica do exercício anterior.
		\begin{enumerate}
		% first column
		\begin{multicols}{2}
			\item $|x| = 3$
			\item $|3| = x$
			\item $|x| = 3$
			\item $|-4| = x$
			\item $|x-1| = \frac{2}{3}$
			\item $-|2x -3| = 2$
			\item $|-x| = 1$
			\item $|2x| = x$
			\item $1-x = |2-8|$
			\item $|1-y| = 1$
			\item $|3-y| = 2y$
			\item $|y+1| = |1-5|$
			\item $|y|^2 = 1$
			\item $|y|^2 = 3^2$
			\item $|y^3| = |11 - 3|$
			\item $|1 - \frac{x}{2}| = |x|$
		\end{multicols}
		\end{enumerate}

		\item Tente resolver as próximas equações sem efetuar as multiplicações. Dica: lembre que para uma multiplicação dar zero, um dos fatores tem que ser zero.
		\begin{enumerate}
		% first column
		\begin{multicols}{2}
			\item $x^2=0$
			\item $x^8=0$
			\item $x(x-1)=0$
			\item $x^5(2x+2)=0$
			\item $x^2(x-5)=0$
			\item $(x-1)^2=0$
			\item $(x+4)^3=0$
			\item $(2x-\frac{1}{8})^4=0$
			\item $(x-5)^18=0$
			\item $(x-1)(x-2)=0$
			\item $2(x-1)x=0$
			\item $x(x-3)^3 = 0$
			\item $x^4(x-2)^6 = 0$
			\item $\frac{3}{5}(x-2)(2x-1)(x+6)=0$
			\item $(x+9)^3(x-1)^5(2x+3)^7=0$
		\end{multicols}
		\end{enumerate}


		\item De novo, tente resolver as próximas equações sem calcular as potências.
		Dica: tente descobrir primeiro que valores o parênteses pode ter.
		Por exemplo, na equação 
		\begin{equation}(19x+5)^2 = 1 \end{equation}
		sei que $(19x+5)$ tem que ser 1 ou $-1$, já que só esses números ao quadrado dão 1. Experimente substituir o parênteses
		por outro símbolo, por exemplo, na equação acima, se defino $(19x+5)=\heartsuit$, posso escrever $\heartsuit^2 = 1$, para o que as soluções são
		$\heartsuit = 1$ e $\heartsuit = -1$. A partir daí, tenho que procurar os casos em que o parênteses ($\heartsuit$) pode ter algum desses valores,
		ou seja, todos os valores de $x$ para os quais $19x+5=1$ OU $19x+5=-1$.
			
		\begin{enumerate}
		% first column
		\begin{multicols}{2}
			\item $x^2=4$
			\item $x^3 = 11^3$
			\item $x^6=64$
			\item $(2x+1)^2=9$
			\item $\left(\frac{x}{3}+\frac{1}{2}\right)^3=1000$
			\item $(x-1)^6=64$
			\item $(x+3)^2=100$
		\end{multicols}
		\end{enumerate}

		\item  De novo, tente resolver as próximas equações sem calcular as potências.
		\begin{enumerate}
		% first column
		\begin{multicols}{2}
			\item $(x-2)^2=(3x+1)^2$
			\item $\left(x+\frac{1}{2}\right)^2=(2x)^2$
			\item $\left(2x+\frac{5}{2}\right)^3=(3x-1)^3$
			\item $(x-1)^5 = (2x)^5$
		\end{multicols}
		\end{enumerate}

		\item Execute as multiplicações, colocando o resultado como soma:
		\begin{enumerate}
			\item exemplo: $(a+b)(c+d +e) = ac + ad + ae + bc + bd+be$
			\item exemplo: $(x+2)^2 = (x+2)(x+2) = x^2+2x + 2x+4 = x^2 + 4x+4$
		\begin{multicols}{2}
			\item $(x-3)^2=$
			\item $(x-2)^2=$
			\item $(x-5)^2=$
			\item $(x+4)^2=$
			\item $2(x+3)^2=$
			\item $3(x+10)^2=$
			\item $(x-1)(x+1)=$
			\item $(x-\sqrt{2}-)(x+\sqrt{2})=$
			\item $(y-2)(y+2)=$
			\item $(a-b)(a+b)=$
			\item $x(x-3)=$
			\item $(x+3)(x-1)=$
			\item $(x-1)^2(2+x)=$
			\item $\frac{3}{5}(x-2)(2x-1)(x+6)=$
			\item $x^5(2x+2)=$
			\item $(x-1)(x-2)=$
			\item $2(x-1)x=$
			\item $x(x-3)^3 = $
			\item $(x-y)(y-x)=$
			\item $(a-b)^3=$
			\item $(x-2)^3=$
			\item $(x-1)(x+a-1)=$
		\end{multicols}
		\end{enumerate}
		
		\item Faça o contrário do exercício anterior: reduza as expressões a seguir a produtos 
		como os dos exercícios acima.
		\begin{enumerate}
			\item exemplo: $x^2 + 2x + 1 = (x+1)^2$
			\item exemplo: $x^3 + 4x^2 + 4x = x(x^2 + 4x + 4) = x(x+2)^2$
		\begin{multicols}{2}
			\item $x^2 -8x + 16=$
			\item $x^2 -6x + 9=$
			\item $x^2 +10x + 25=$
			\item $2x^2 -4x + 2=$
			\item $y^2 - x^2=$
			\item $3x^3 + 12x^2 + 12x=$
			\item $x^2 + 4ax + 4a^2=$
			\item $x^3 + 3x^2 + 3x + 1=$
		\end{multicols}
		\end{enumerate}

\end{enumerate}
	\end{document}
